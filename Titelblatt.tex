\documentclass[
    toc=bibliographynumbered,
    a4paper,
    11pt,
    twoside=false,]{article}


\usepackage[utf8]{inputenc}     
\usepackage[T1]{fontenc}        
\usepackage[ngerman]{babel}  
\usepackage{fontspec}   
\usepackage{geometry}   
\usepackage{titlesec}  
\usepackage{setspace}
\usepackage{fancyhdr} 
\usepackage{lipsum}
\usepackage{wrapfig}
\usepackage{csquotes}
\usepackage{siunitx}
\usepackage{amsmath}
\usepackage[font=it]{caption}
\usepackage{float}
\setlength{\parindent}{0pt}
\usepackage[backend=biber, style=phys, biblabel=brackets, block=ragged, language=autobib, autolang=other]{biblatex}
\addbibresource{literatur.bib}
%\usepackage{hyperref}
\usepackage{amssymb}



\fancyhf{}
\fancyfoot[C]{\thepage}
\thispagestyle{empty}
\fancyhead[L]{Gruppe 04}
\fancyhead[C]{\Versuchnummer {} \Versuch}
\fancyhead[R]{\Abgabedatum}

\geometry{left=3cm, right=3cm, top=3cm, bottom=3cm}

\usepackage{xcolor}             
\usepackage{graphicx}           


\usepackage{ragged2e}         
\RaggedRight


\newcommand{\Versuchnummer}{W6}                           %CHange iiit
\newcommand{\Versuch}{Die Dampfdruckkurve von Wasser}          %Change iiit
\newcommand{\Abgabedatum}{30.11.2025}                           %Change iiit
\newcommand{\Versuchsdatum}{19.11.2025}              %Change iiit






\newcommand{\sectionstyle}[1]{\color{teal!40!gray}\bfseries\LARGE #1}         
\newcommand{\subsectionstyle}[1]{\color{teal!40!gray}\bfseries\Large #1}
\newcommand{\subsubsectionstyle}[1]{\color{teal!40!gray}\bfseries\small #1}
\titleformat{\section}{\sectionstyle}{\thesection}{1em}{}
\titleformat{\subsection}{\subsectionstyle}{\thesubsection}{1em}{}
\titleformat{\subsubsection}{\subsubsectionstyle}{\thesubsubsection}{1em}{}











\begin{document}



\begin{titlepage}
    
    {\color{teal!40!gray}\fontsize{40}{30}\selectfont\bfseries Physikalisches\\ Anfängerpraktikum}\\
    \vspace{0,5cm}
    {\LARGE {\textbf {Universität Augsburg\\Wintersemester 2025/26}}\par}
    \vspace{3cm}
    

   {\LARGE \textbf {Versuch: {}\Versuchnummer {} \Versuch  }}\\ 
   \vspace{1cm}


     \begin{minipage}{0.7\textwidth}
        {\large \setstretch{2}{Gruppe: \hspace{2,8cm} G 04}}\\ \vspace{0,3cm}   
    {\large \setstretch{2}{Versuchsdatum: \hspace{1,55cm}\Versuchsdatum}}\\\vspace{0,3cm}
     {\large \setstretch{2}{Abgabedatum: \hspace{1,8cm}\Abgabedatum}}\\\vspace{1,5cm}
    {\large \setstretch{2}{Gemeinsames Versuchsprotokoll}}\\
     {\large \setstretch{2}{Ferdinand Frey\\Tom Glaser}}

  \end{minipage}
     

\vspace{2cm}
    \begin{minipage}{0.5\textwidth}
            \includegraphics[width =\linewidth]{Bilder/Logo.jpg}
    \end{minipage}\hfill

%\begin{figure}[H]
 %   \centering
%\includegraphics[width=0.4\textwidth, height = 4cm, scale=3.0]{Bilder/Titelbild W-6.png}%Bild ändern
%\caption{Darstellung des Versuchsaufbau \cite{Titelbild}}   
%\end{figure}      
    \vfill
\end{titlepage}
\setcounter{page}{2}


\pagestyle{fancy}
\markboth{INHALTSVERZEICHNIS}{}
\tableofcontents
\newpage

\pagestyle{fancy}
\section{Einleitung}

Die Phasenübergänge von Wasser sind ein wichtiger Bestandteil 
unseres täglichen Lebens und somit auch der Phasenübergang 
Wassers von Flüssig zu Gasförmig. Dieser Phasenübergang findet 
sich im Alltag, in der Technik und in der Forschung. In der 
Technik kann er dazu verwendet werden um Mikro-Prozessoren zu 
kühlen, diese können auf sehr kleinem Raum viel Wärme abgeben, 
damit sie dann immernoch effizient laufen können werden 
Wärmerohre verwendet die diese entstehende Wärme abtransportieren 
können. Die Rohre funktionieren auf der Basis dass eine 
Flüssigkeit auf der einen Seite verdampft und auf der anderen 
Seite wieder kondensiert und somit die Verdampfungsenergie zum 
zweiten Punkt bringt. Damit nun aber nicht der Verdampfunspunkt 
wie zum Beispiel bei Wasser nicht bei 100°C liegt, wird der 
Druck innerhalb des Rohres gesenkt und der Verdampfungspunkt 
sinkt ebenso.  (Quelle)

Der Versuch ist dem Bereich der Wärmelehre zuzuorden.
\newpage
\section{Theoretische Grundlagen}

\subsection{Wasserstrahlpumpe}
Die im Versuch verwendete Wasserstrahlpumpe funktioniert nicht 
wie eine Verdrängungspumpe, bei der Verdrängunspumpe gibt es 
Hauptsächlich vier Arbeitsschritte. 1. Ansaugen 2. Transportieren 
und Verdichten 3. Ausschieben. Im ersten Schritt wird aus dem 
zu entleerenden Volumen Luft angesaugt, die daraufhin im 
zweiten Schritt verdichtet und zum Ausgabepunkt transportiert 
wird, dort wird sie dann ausgeschoben, daraufhin wiederholt 
sich der Prozess nun. Bei der Wasserstrahlpumpe im Gegensatz 
beruht die Funktionsweise auf der Kontinuitätsgleichung. Bei 
der Wasserstrahlpumpe fließt Wasser erst durch einen Schlauch 
der verengt wird, somit erhöht sich die Geschwindigkeit des 
Wassers und der Druck nimmt ab. Danach wird der Schlauch 
schlagartig wieder größer wodurch dann Luft aus der Umgebung 
angesaugt wird, in das Wasser eingelagert wird und somit auch 
Abtransportiert. 
\begin{figure}[H]
    \centering
    \includegraphics[height = 10cm]{Bilder/Wasserstrahlpumpe.png}
    \caption{Eine Schematische Darstellung der Wasserstrahlpumpe\cite{Wasserstrahlpumpe}}
    
    
\end{figure}
\newpage

\subsection{Phasendiagramm}
Die drei Phasen die im alltäglichem Leben beobachtet werden 
können sind, Gasförmig, Flüssig und Fest. Im Gasförmigen Zustand 
haben die jeweiligen Teilchen genug Energie um Wechselwirkungen 
zwischeneinander zu durchbrechen und sich frei im Raum zu 
bewegen. In der Flüssigen Phase haben die Teilchen nicht mehr 
genug Energie um die gegenseitigen Wechselwirkungne zu 
überwinden, sie haben aber immernoch genug Energie übrig um 
sich innerhalb der Flüssigkeit frei zu bewegen. In der festen 
Phase haben die Teilchen kaum noch Energie übrig weswegen sie 
sich in einem festen Gitter an ordnen und nur noch an ihrem 
festen Platz schwingen können. 
\begin{figure}[H]
\centering
\includegraphics[scale = 1]{Bilder/Phasendiagramm Wasser.png}
\caption{Das p,T-Phasendiagramm von Wasser, in diesem Versuch 
wird die Siedepunktskurve oder auch Dampfdruckkurve gennant 
betrachtet\cite{Phasendiagrammwasser}}
\end{figure}

Die im Versuch zu betrachtende Dampfdruckkurve ist die Verbindung 
zwischen Gasförmig und Flüssig, genau auf der Kurve befinden 
sich die beiden Phasen in einem Thermodynamischen Gleichgewicht 
welches durch das Gibbs Potential ausgedrückt werden kann.
\begin{equation}
    \Delta G = U + p \cdot V - S\cdot T
\end{equation}
Diese Formel gibt das Potential der Teilchen an und kann ebenfalls als,
\begin{equation}
    \Delta G = \Delta \left(U + p \cdot V - S\cdot T\right) = 0
\end{equation}
dargestellt werden. Diese Formel bezeichnet dass das Potential des ,in diesem Fall 
,Wassers immer so niedrig wie möglich ist. Wenn dieses Gleichgewicht, 
also $\Delta G = 0$ nun also auf der Dampfdruckkurve liegt können 
sogar beide Phasen gleichzeitig existieren. Auf dem Triple-Punkt 
existieren sogar alle drei Phasen nebeneinander.

\newpage
\subsection{Flüssigkeitsbarometer}
Flüssigkeitsbarometer bestehen aus einem am Oberen Ende geschlossenem 
Rohr welches bis auf eine kleine Menge an Luft mit einer Flüssigkeit 
gefüllt sind. Da diese Flüssigkeit nun durch die Gravitation 
nach unten gezogen wird entsteht ein Kräftegleichgewicht und 
die Flüssigkeit wird im Rohr gehalten. Damit nun bei Druck Änderungen 
außerhalb des Rohres die Füllhöhe steigen kann wird außerhalb 
ein nach oben geöffnetes Reservoir angebracht damit ein 
Druckausgleich stattfinden kann. Flüssigkeitsbarometer 
verwenden meist Quecksilber als Flüssigkeit da Quecksilber eine 
sehr hohe dichte hat und das Barometer dadurch kompakt gehalten 
werden kann. Außerdem verdunstet Quecksilber nur sehr langsam 
weswegen Quecksilberbarometer sehr lange benutzt werden können 
bevor sie erneuert werden müssen.
\begin{figure}[H]
    \centering
    \includegraphics{Bilder/Flüssigkeitsbarometer.png}
    \caption{Die Schematische Darstellung eines Flüssigkeitsbarometer \cite{Flüssigkeitsbarometer}}
\end{figure}


\newpage
\section{Versuchsbeschreibung}

\subsection{Versuchsaufbau}
Um den Versuch durchzuführen, werden folgende Gegenstände benötigt:
Eine Wasserstrahlpumpe, ein Rundkolben gefüllt mit Wasser,
ein Thermometer, Siedesteinchen, ein U-Rohr-Manometer, ein
Rundkolben als Quecksilber-Reservoir, ein Zwei-Wege-Ventil,
ein Drei-Wege-Ventil, ein kurzes Glasrohr, ein Drei-Wege-Verbinder,
Millimeterpapier, genügend Verbindungsschläuche mit Dichtungen,
um alles passend zu verbinden, sowie ein Laborstativ mit Klemmen,
damit der Versuch übersichtlich aufgebaut werden kann. \

\begin{figure}[H]
\centering
\includegraphics[scale=0.2]{Bilder/Versuchsaufbau W-6.jpg}
\caption{Eine selbst angefertigte Skizze des Aufbaus}
\label{Versuchsaufbau}
\end{figure}

Nun zum Aufbau: Zuerst wird die Wasserstrahlpumpe
$\left(\text{Nummer~1}\right)$ an eine externe Wasserquelle angeschlossen und
mithilfe eines Schlauches an das Drei-Wege-Ventil angebracht. Dieses Drei-Wege-Ventil wird anschließend mit dem Zwei-Wege-Ventil verbunden, das bereits mit einem Glasrohr verbunden ist, damit später Luft kontrolliert in das System eingeleitet werden kann $\left(\text{Nummer~3}\right)$.
Danach wird am Drei-Wege-Ventil der Glaskolben $\left(\text{Nummer~2}\right)$
mit Wasser und Siedesteinchen befestigt, in den Heizpilz
$\left(\text{Nummer~5}\right)$ eingesetzt und das Thermometer positioniert. Anschließend
wird das U-Rohr-Manometer mithilfe des Drei-Wege-Verbinders am Glaskolben befestigt. Zuletzt wird das Quecksilber-Reservoir am U-Rohr angebracht $\left(\text{Nummer~4}\right)$, das Millimeterpapier aufgestellt und das System vollständig abgedichtet.

Der Aufbau sollte nun grob wie in Abbildung \ref{Versuchsaufbau} dargestellt aussehen.

\newpage

\subsection{Versuchsdurchführung}
Bevor der Versuch begonnen werden kann, sind zunächst einige Vorbereitungen erforderlich. Der Druck im System muss reduziert werden. Hierzu wird das Drei-Wege-Ventil in alle Richtungen geöffnet, während das Zwei-Wege-Ventil geschlossen bleibt. Danach wird die Wasserstrahlpumpe eingeschaltet, bis sich die Anzeige des Quecksilbermanometers nicht mehr verändert. Sobald dieser stationäre Zustand erreicht ist, wird das Drei-Wege-Ventil zur Wasserstrahlpumpe geschlossen.
Im nächsten Schritt werden die Temperatur des Wassers sowie die Höhe des Quecksilberbarometers gemessen. Daraufhin wird der Rundkolben mithilfe des Heizpilzes erhitzt, bis das Wasser zu sieden beginnt. Um die Messwerte reproduzierbar zu halten, ist es notwendig, einen konsistenten Siedepunkt festzulegen. In diesem Versuch wurde der Siedepunkt als der Zeitpunkt definiert, an dem mehrere gleichmäßige Blasenströme im Rundkolben aufsteigen.
Sobald dieser definierte Siedepunkt erreicht ist, werden erneut die Temperatur des Wassers und die Höhe des Barometers erfasst. Anschließend wird der Druck im System leicht erhöht: Dazu wird das Glasrohr an der in Abbildung \ref{Versuchsaufbau} markierten Position kurz mit einem Finger verschlossen, das Zwei-Wege-Ventil vorsichtig geöffnet und wieder geschlossen und erst danach der Finger entfernt. Dieser Vorgang wird zwei- bis dreimal wiederholt, um einen ausreichend großen Druckunterschied zu erzeugen.
Nach dieser Prozedur kann die Temperatur weiter erhöht werden, und die Messschritte werden wiederholt, bis insgesamt etwa 20 bis 25 Messwerte aufgenommen wurden.
\newpage
\section{Auswertung}
\subsection{Umgebungsdruck}
Für die Quecksilbersäule wurde eine Höhe von 953,0~mbar und für die Kuppel eine Höhe von 1,2~mbar gemessen. Die Raumtemperatur betrug $(21,1 \pm 0,1)~^\circ\mathrm{C}$. 
Für die geographische Lage von Augsburg wurden eine Höhenlage von 500~m, eine östliche Länge von 10°\,52' und eine nördliche Breite von 48°\,21' verwendet. 
Mit diesen Werten lässt sich die gemessene Quecksilbersäule entsprechend korrigieren.


\begin{table}[H]
\centering
\begin{tabular}{c|c}

Gemessener Druck & 953,0mb \\
\hline
Temperaturkorrektur(21°C) & $- 3,64$mb \\

Kuppenkorrektur(1,2mb) & +0,64mb\\

Höhenkorrektur(500m) &  $ -0,09$mb \\

Breitenkorrektur(48°) &  $ +0,39$mb \\
\hline
Korregierter Druck $p_0$ & = 950,30mb \\

\end{tabular}
\caption{korrigierter Druck in mbar}
\end{table}


Damit ergibt sich ein wahrer Atmosphärendruck von $p_0 = 950,30~\mathrm{mbar} = 95030~\mathrm{Pa}$. 
Hier wird angenommen, dass keine weiteren Fehler aufgetreten sind, und daher wird auf eine Fehlerrechnung verzichtet. 
Die Steighöhe der Quecksilbersäule ist im Folgenden angegeben.


\begin{table}[H]
\centering
\begin{tabular}{c|c}
$h_\mathrm{Hg}$ [mm] & $T$ [K] \\
\hline
645 $\pm$ 3 & 315.35 $\pm$ 0.05 \\
627 $\pm$ 3 & 324.05 $\pm$ 0.05 \\
590 $\pm$ 3 & 330.05 $\pm$ 0.05 \\
581 $\pm$ 3 & 331.85 $\pm$ 0.05 \\
555 $\pm$ 3 & 334.85 $\pm$ 0.05 \\
548 $\pm$ 3 & 336.65 $\pm$ 0.05 \\
536 $\pm$ 3 & 337.65 $\pm$ 0.05 \\
522 $\pm$ 3 & 339.95 $\pm$ 0.05 \\
508 $\pm$ 3 & 341.35 $\pm$ 0.05 \\
490 $\pm$ 3 & 343.45 $\pm$ 0.05 \\
473 $\pm$ 3 & 345.15 $\pm$ 0.05 \\
453 $\pm$ 3 & 348.15 $\pm$ 0.05 \\
438 $\pm$ 3 & 349.15 $\pm$ 0.05 \\
420 $\pm$ 3 & 351.25 $\pm$ 0.05 \\
400 $\pm$ 3 & 352.35 $\pm$ 0.05 \\
373 $\pm$ 3 & 354.35 $\pm$ 0.05 \\
350 $\pm$ 3 & 356.15 $\pm$ 0.05 \\
328 $\pm$ 3 & 357.45 $\pm$ 0.05 \\
305 $\pm$ 3 & 359.15 $\pm$ 0.05 \\
290 $\pm$ 3 & 359.85 $\pm$ 0.05 \\
260 $\pm$ 3 & 361.15 $\pm$ 0.05 \\
238 $\pm$ 3 & 362.15 $\pm$ 0.05 \\
170 $\pm$ 3 & 365.85 $\pm$ 0.05 \\
\end{tabular}
\caption{Die Messwerte von der Quecksilbersäule und der zugehörigen Temperatur mit Ablesefehlern}
\label{tb:messwerte}
\end{table}

gegen die entsprechende Temperatur aufgetragen. Bei dieser Messung gibt es einige Fehlerquellen. 
Die wohl größte ist die Bestimmung des Zeitpunkts, an dem das Wasser zu kochen beginnt. 
Dieser wurde bei der Durchführung als der Zeitpunkt definiert, an dem am gesamten Rand des Gefäßes Luftblasen zu sehen waren, sich also an mehreren Siedesteinchen Blasen bilden. 
Trotz dieser Definition muss dieser Zeitpunkt immer noch subjektiv festgestellt werden. 
Die Temperatur und der Druck müssen dann zeitgleich abgelesen werden. Beides ist fehleranfällig. 
Eine weitere Fehlerquelle ist das Ablesen der Höhe der Quecksilbersäule. Hier wird zum einen die Höhe der Kuppel nicht berücksichtigt, zum anderen muss die Höhe an einer Millimeterskala abgelesen werden, was ebenfalls fehleranfällig ist. 
Aus diesen Gründen beträgt der Messfehler der Höhe $\pm 3~\mathrm{mm}$. 
Der Fehler der Temperatur entsteht dadurch, dass das Display nur eine Nachkommastelle anzeigen kann.\\
Durch die Differenz zwischen dem Außendruck und dem Druck im Gefäß lässt sich der Dampfdruck $p_D$ betrachten. 
Betrachtet man jedoch zuerst den Druck $p$.


\begin{equation}
p = \frac{F}{A}
\end{equation}

Da die relevante Kraft hier die Gravitationskraft ist, folgt mit $F = F_g$ daraus:

\begin{equation}
p = \frac{m_{Hg} \cdot g}{A_{Rohr}} = \frac{\rho_{Hg} \cdot A_{Rohr} \cdot h \cdot g}{A_{Rohr}} = \rho_{Hg} \cdot g \cdot h = p_0 - p_D
\end{equation}

Damit folgt für den Dampfdruck.
\begin{equation}
p_D = p_0 - p_{Hg} \cdot g \cdot h
\label{eq:d_P2_FR}
\end{equation}

Der Fehler ist gegeben durch.

\begin{equation}
\Delta p_D = \pm \rho_{Hg} \cdot g \cdot \Delta h
\label{eq:d_P2}
\end{equation}

Mit dieser Formel kann der Dampfdruck $P_D$ berechnet werden. Für die Erdbeschleunigung wurde $g = 9,81 \frac{\text{m}}{\text{s}^2}$ verwendet. Für den Außendruck $p_0 = 95030$ Pa. $P_{Hg}= 13534 \frac{\text{kg}}{\text{m}^3}$ ist ein Literaturwert \cite{Hgdichte}. In Tabelle \ref{tb:pd_K} sind die daraus Folgenden Dampfdrücke aufgetragen.


\begin{table}[H]
\centering
\begin{tabular}{r | r}
T [K] & $P_D$ [Pa] \\
\hline
315.35 $\pm$ 0.05 & 9318.4 $\pm$ 400 \\
324.05 $\pm$ 0.05 & 11710.3 $\pm$ 400 \\
330.05 $\pm$ 0.05 & 16627.1 $\pm$ 400 \\
331.85 $\pm$ 0.05 & 17823.1 $\pm$ 400 \\
334.85 $\pm$ 0.05 & 21278.1 $\pm$ 400 \\
336.65 $\pm$ 0.05 & 22208.3 $\pm$ 400 \\
337.65 $\pm$ 0.05 & 23803.0 $\pm$ 400 \\
339.95 $\pm$ 0.05 & 25663.4 $\pm$ 400 \\
341.35 $\pm$ 0.05 & 27523.8 $\pm$ 400 \\
343.45 $\pm$ 0.05 & 29915.7 $\pm$ 400 \\
345.65 $\pm$ 0.05 & 32174.8 $\pm$ 400 \\
348.15 $\pm$ 0.05 & 34832.5 $\pm$ 400 \\
349.15 $\pm$ 0.05 & 36825.8 $\pm$ 400 \\
351.25 $\pm$ 0.05 & 39217.8 $\pm$ 400 \\
352.35 $\pm$ 0.05 & 41875.5 $\pm$ 400 \\
354.35 $\pm$ 0.05 & 45463.4 $\pm$ 400 \\
357.15 $\pm$ 0.05 & 48519.8 $\pm$ 400 \\
357.45 $\pm$ 0.05 & 51443.3 $\pm$ 400 \\
359.15 $\pm$ 0.05 & 54499.7 $\pm$ 400 \\
359.85 $\pm$ 0.05 & 56493.0 $\pm$ 400 \\
361.15 $\pm$ 0.05 & 60479.6 $\pm$ 400 \\
362.15 $\pm$ 0.05 & 63403.1 $\pm$ 400 \\
365.85 $\pm$ 0.05 & 72439.3 $\pm$ 400 \\
\end{tabular}
\caption{Die mit Hilfe von Gleichung (\ref{eq:d_P2_FR}) und den Werten aus Tabelle \ref{tb:messwerte} erechneten Dampfdrücke mit Fehlern gemäß Gleichung (\ref{eq:d_P2})}
\label{tb:pd_K}
\end{table}


Wenn man diese Werte, wie im folgenden graphisch aufträgt,

\begin{figure}[H]
    \centering
    \includegraphics[height = 12cm]{Bilder/pDgegenT.png}
    \caption{Der Dampfdruck $P_D$ aufgetragen gegen die Temperatur T die Fitkurve wurde mittels Numpy.polyfit erstellt.}
    \label{gl:5}
\end{figure}

ist der exponentielle Zusammenhang, so wie  in Gleichung (\ref{eq:pd}) verausgesagt, zwischen Dampfdruck und Temperaturanstieg zu erkennen. Vergleicht man dies mit den theoretischen Werten für den Siedepunkt von 100°C bei Normaldruck, so erhält man mit Hilfe der Clausius-Clapeyron-Gleichung

\begin{equation}
p_D(T) = 101300~\mathrm{Pa} \cdot \exp \Bigg[ - \frac{40.642~\mathrm{kJ/mol}}{8.314~\mathrm{J/(mol \cdot K)}} \left( \frac{1}{T} - \frac{1}{373.15~\mathrm{K}} \right) \Bigg]
\end{equation}

Wenn man diese Werte in Abbildung \ref{gl:5} einträgt,

\begin{figure}[H]
    \centering
    \includegraphics[height = 12cm]{Bilder/Lieteraturwerte.png}
    \caption{Der Dampfdruck $P_D$ und der Theortische Druck aus Gleichung (13) aufgetragen gegen die Temperatur T die Fitkurve wurde mittels Numpy.polyfit erstellt.}
    
\end{figure}

kann man eine leichte Abweichung von den Messwerten, die mit steigenden Temperaturen zunimmt, beobachtet werden. Ein Grund könnte möglicherweise sein, dass die Steighöhe der Quecksilbersäule von der Temperatur oder Druck im System, noch wegen anderen Gründen abhängig ist.


\newpage
\subsection{Lineare Dampfdruckkurve}
Mit Hilfe des Logarithmus lässt sich ein Linearer Zusammenhang zwischen $P_D$ und $1/T$ herstellen. 

\begin{equation}
\ln(p_D) = \ln(p_D^{0}) 
- \frac{r}{R}\left( \frac{1}{T} - \frac{1}{T_0} \right)
= -\,\frac{r}{R}\cdot \frac{1}{T}
+ \frac{r}{R T_0}
+ \ln(p_D^{0}).
\label{eq:ln}
\end{equation}

Trägt man die Werte graphisch auf so errhält man Abbildung \ref{gf:ln}

\begin{figure}[H]
    \centering
    \includegraphics[height = 12cm]{Bilder/Reziproke.png}
    \caption{Der Logarithmus Dampfdruck $ln(P_D)$ aufgetragen gegen die Temperatur T in $\frac{1}{K}$ die Fitkurve wurde mittels Numpy.polyfit erstellt.}
    \label{gf:ln}
\end{figure}

Die Steigungen sind in Tabelle 4 aufgelistet, die Einheit der Steigung ist in [K] da der Logarithmus dimensionslos ist.

\begin{table}[H]
\centering
\begin{tabular}{r | r}
Gerade & Steigung [K] \\
\hline
Ausgleichsgerade & -4820 \\
Grenzgerade 1 & -4525\\
Grenzgerade 2 & -5133 \\

\end{tabular}
\caption{Die Steigung der Geraden} 
\end{table}

Damit kann man den Fehler der Steigung folgendermaßen berechnen:
\begin{equation}
\Delta m \pm \frac{m_2 - m_1}{2} = 304 \text{K}
\end{equation}

\subsection{Molare Verdampfungswärme}

Durch Differentiation und Umstellung von Gleichung (\ref{eq:ln}) ergibt sich für die molare Verdampfungswärme unter Verwendung der allgemeinen Gaskonstante $R = 8.314 \,\frac{\mathrm{J}}{\mathrm{mol\,K}}$

\begin{equation}
r = -m \cdot R \approx 40.07 \frac{\mathrm{KJ}}{\mathrm{mol}}
\end{equation}

Da $R$ als fehlerfrei angenommen wird, ergibt sich für den Größtfehler von $r$ folgende Gleichung:

\begin{equation}
\Delta r = \pm \left| \frac{\partial r}{\partial m} \right| \Delta m = \pm \Delta m \cdot R
\end{equation}

Nach Einsetzen der Werte ergibt sich aus den Messwerten die molare Verdampfungswärme für Wasser zu:

\begin{equation}
r \approx (40,07 \pm 2,47)~\mathrm{kJ/mol}
\end{equation}

Der Literaturwert aus \cite{Tipler} beträgt $r_L = 40,66~\mathrm{kJ/mol}$. 
Somit liegt die bestimmte Verdampfungswärme $r$ innerhalb der Fehlerschranke, die durch $\Delta r$ angegeben ist.
\newpage
\section{Zusammenfassung}

Die zentrale Erkenntnis des Versuchs, die Abhängigkeit des Siedepunkts von Wasser vom Umgebungsdruck sowie von der Temperatur, konnte bestätigt werden. Die gemessene Dampfdruckkurve stimmt weitestgehend mit der theoretischen Dampfdruckkurve überein. Auch die materialspezifische Konstante der molaren Verdampfungswärme von Wasser, welche z.B. auch für den Versuch über die Siedepunkterhöhung wichtig ist, konnte nahe am Literaturwert bestimmt werden. Die Abweichungen von den Literaturwerten können auf verschiedene Fehlerquellen zurückgeführt werden: Zum einen die Subjektivität bei der Bestimmung des Siedepunkts, zum anderen Ablesefehler bei der Quecksilbersäule. Letzteres könnte mit einem genaueren Messgerät für den Druck behoben werden. Um den Siedepunkt objektiv zu messen, ist wahrscheinlich ein anderer Versuchsaufbau erforderlich.
\newpage
\section{Anhang}
\begin{figure}[H]
    \centering
    \includegraphics[height = 20cm]{Bilder/Unbenannt.jpg}
    
\end{figure}

\newpage
\input{Literaturverzeichnis}



\end{document}
