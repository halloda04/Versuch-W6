\section{Versuchsbeschreibung}

\subsection{Versuchsaufbau}
Um den Versuch durchzuführen, werden folgende Gegenstände benötigt:
Eine Wasserstrahlpumpe, ein Rundkolben gefüllt mit Wasser,
ein Thermometer, Siedesteinchen, ein U-Rohr-Manometer, ein
Rundkolben als Quecksilber-Reservoir, ein Zwei-Wege-Ventil,
ein Drei-Wege-Ventil, ein kurzes Glasrohr, ein Drei-Wege-Verbinder,
Millimeterpapier, genügend Verbindungsschläuche mit Dichtungen,
um alles passend zu verbinden, sowie ein Laborstativ mit Klemmen,
damit der Versuch übersichtlich aufgebaut werden kann. \

\begin{figure}[H]
\centering
\includegraphics[scale=0.2]{Bilder/Versuchsaufbau W-6.jpg}
\caption{Eine selbst angefertigte Skizze des Aufbaus}
\label{Versuchsaufbau}
\end{figure}

Nun zum Aufbau: Zuerst wird die Wasserstrahlpumpe
$\left(\text{Nummer~1}\right)$ an eine externe Wasserquelle angeschlossen und
mithilfe eines Schlauches an das Drei-Wege-Ventil angebracht. Dieses Drei-Wege-Ventil wird anschließend mit dem Zwei-Wege-Ventil verbunden, das bereits mit einem Glasrohr verbunden ist, damit später Luft kontrolliert in das System eingeleitet werden kann $\left(\text{Nummer~3}\right)$.
Danach wird am Drei-Wege-Ventil der Glaskolben $\left(\text{Nummer~2}\right)$
mit Wasser und Siedesteinchen befestigt, in den Heizpilz
$\left(\text{Nummer~5}\right)$ eingesetzt und das Thermometer positioniert. Anschließend
wird das U-Rohr-Manometer mithilfe des Drei-Wege-Verbinders am Glaskolben befestigt. Zuletzt wird das Quecksilber-Reservoir am U-Rohr angebracht $\left(\text{Nummer~4}\right)$, das Millimeterpapier aufgestellt und das System vollständig abgedichtet.

Der Aufbau sollte nun grob wie in Abbildung \ref{Versuchsaufbau} dargestellt aussehen.

\newpage

\subsection{Versuchsdurchführung}
Bevor der Versuch begonnen werden kann, sind zunächst einige Vorbereitungen erforderlich. Der Druck im System muss reduziert werden. Hierzu wird das Drei-Wege-Ventil in alle Richtungen geöffnet, während das Zwei-Wege-Ventil geschlossen bleibt. Danach wird die Wasserstrahlpumpe eingeschaltet, bis sich die Anzeige des Quecksilbermanometers nicht mehr verändert. Sobald dieser stationäre Zustand erreicht ist, wird das Drei-Wege-Ventil zur Wasserstrahlpumpe geschlossen.
Im nächsten Schritt werden die Temperatur des Wassers sowie die Höhe des Quecksilberbarometers gemessen. Daraufhin wird der Rundkolben mithilfe des Heizpilzes erhitzt, bis das Wasser zu sieden beginnt. Um die Messwerte reproduzierbar zu halten, ist es notwendig, einen konsistenten Siedepunkt festzulegen. In diesem Versuch wurde der Siedepunkt als der Zeitpunkt definiert, an dem mehrere gleichmäßige Blasenströme im Rundkolben aufsteigen.
Sobald dieser definierte Siedepunkt erreicht ist, werden erneut die Temperatur des Wassers und die Höhe des Barometers erfasst. Anschließend wird der Druck im System leicht erhöht: Dazu wird das Glasrohr an der in Abbildung \ref{Versuchsaufbau} markierten Position kurz mit einem Finger verschlossen, das Zwei-Wege-Ventil vorsichtig geöffnet und wieder geschlossen und erst danach der Finger entfernt. Dieser Vorgang wird zwei- bis dreimal wiederholt, um einen ausreichend großen Druckunterschied zu erzeugen.
Nach dieser Prozedur kann die Temperatur weiter erhöht werden, und die Messschritte werden wiederholt, bis insgesamt etwa 20 bis 25 Messwerte aufgenommen wurden.