\section{Versuchsbeschreibung}

\subsection{Versaufbau}
Um den Versuch Durchzuführen werden folgende Gegenstände benötigt. 
Eine Wasserstrahlpumpe, einen Rundkolben gefüllt mit Wasser
ein Thermometer, Siedesteinchen, ein U-Rohr-Manometer, ein 
Rundkolben als Quecksilber-Reservoir, ein Zwei-Wege-Ventil,  
ein Drei-Wege-Ventil, ein kurzes Glasrohr, ein drei Wege Verbinder, 
Millimeter-Papier, genug Verbindungsschläuche mit Dichtungen 
um alles passend zu verbinden und ein Laborstativ mit Klemmen 
damit der Versuch sortiert werden kann. \\
\begin{figure}[H]
    \centering
    \includegraphics[scale = 0.2]{Bilder/Versuchsaufbau W-6.jpg}
    \caption{Eine Selbst angefertigte Skizze vom Aufbau}
    \label{Versuchsaufbau}
\end{figure}
Nun zum Aufbau. Zuerst wird die Wasserstrahlpumpe
$\left(\text{Nummer~1}\right)$ an eine externe Wasserquelle angeschlossen und 
mithilfe eines Schlauches an dem drei Wege Ventil befestigt, dieses drei 
Wege Ventil wird nun mit dem Zwei wegeventil Verbunden welches schon mit 
einem Glasrohr Verbunden wurde, damit später wieder Luft kontrolliert 
in das System gelassen werden kann $\left(\text{Nummer~3}\right)$. 
Nun wird wieder beim Drei-Wege-Ventil, der Glaskolben $\left(\text{Nummer~2}\right)$ 
mit dem Wasser und den Siedesteinchen befestigt, in den Heizpilz 
$\left(\text{Nummer~5}\right)$ gesetzt und das Thermometer eingesetzt. Anschließend 
wird das U-Rohr-Manometer mithilfe der Drei-Wege-Verbinder am Glaskolben 
befestigt. Zuletzt wird das Quecksilber Reservoir am U-Rohr angebracht,
$\left(\text{Nummer~4}\right)$ das Millimeter Papier aufgestellt und alles abgedichtet. 
Der Aufbau sollte nun Grob wie die Abbildung \ref{Versuchsaufbau} aussehen.

\newpage

\subsection{Versuchsdurchführung}
Bevor der Versuch Anfangen kann müssen noch vorbereitungen getroffen werden. 
Hierzu muss der Druck im System verringert werden. Dafür wird das 
Drei-Wege-Ventil in alle Richtungen geöffnet und das Zwei-Wege-Ventil 
wird geschlossen, nun wird die Wasserstrahlpumpe angeschaltet und gewartet 
bis sich das Quecksilber-Manometer nicht mehr ändert, sobald dieser 
Zustand Eintritt wird das Drei-Wege-Ventil zur Wasserstrahlpumpe geschlossen. 
Zuerst wird nun die Temperatur gemessen und die Höhe des Quecksilberbarometers 
gemessen. Daraufhin wird der Rundkolben mithilfe des Heizpilzes erhitzt 
und gewartet bis das Wasser anfängt zu sieden. Jetzt ist erstmal ein 
wichtiger Einschub notwendig, um die in diesem Protokoll verwendeten Werte 
zu reproduzieren muss der gleiche Siede Zeitpunkt gewählt werden. In unserem 
Fall wurde der Zeitpunkt gewählt an dem mehrere Gleichmäßige Ströme an 
Blasen im Rundkolben aufgestiegen sind. Sobald das Wasser nun diesen 
gewählten Siedepunkt erreicht hat wird wieder die Temperatur des Wasser, 
und die Höhe des Barometers gemessen, der letzte Schritt dieses 
Messungskreislaufes ist den Druck im System ein klein Wenig zu erhöhen 
dafür wird nun bei Punkt 3 der Abbildung \ref{Versuchsaufbau} das Glasrohr 
mithilfe eines Fingers verschlossen, das Zwei wege Ventil Vorsichtig 
geöffnet und wieder geschlossen und erst danach der Finger wieder vom 
Glasrohr herunter genommen. Dieser Prozess wird zwei bis drei mal wiederholt 
sodass es ein hoch genugen Druckunterschied gibt. Nun kann die Temperatur 
weiter erhöht werden und diese Schritte werden wiederholt bis man so 
20-25 Messungen hat.

