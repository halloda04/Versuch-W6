\section{Theoretische Grundlagen}

\subsection{Wasserstrahlpumpe}
Die im Versuch verwendete Wasserstrahlpumpe funktioniert nicht 
wie eine Verdrängungspumpe, bei der Verdrängunspumpe gibt es 
Hauptsächlich vier Arbeitsschritte. 1. Ansaugen 2. Transportieren 
und Verdichten 3. Ausschieben. Im ersten Schritt wird aus dem 
zu entleerenden Volumen Luft angesaugt, die daraufhin im 
zweiten Schritt verdichtet und zum Ausgabepunkt transportiert 
wird, dort wird sie dann ausgeschoben, daraufhin wiederholt 
sich der Prozess nun. Bei der Wasserstrahlpumpe im Gegensatz 
beruht die Funktionsweise auf der Kontinuitätsgleichung. Bei 
der Wasserstrahlpumpe fließt Wasser erst durch einen Schlauch 
der verengt wird, somit erhöht sich die Geschwindigkeit des 
Wassers und der Druck nimmt ab. Danach wird der Schlauch 
schlagartig wieder größer wodurch dann Luft aus der Umgebung 
angesaugt wird, in das Wasser eingelagert wird und somit auch 
Abtransportiert. 
\begin{figure}[H]
    \centering
    \includegraphics[height = 10cm]{Bilder/Wasserstrahlpumpe.png}
    \caption{Eine Schematische Darstellung der Wasserstrahlpumpe\cite{Wasserstrahlpumpe}}
    
    
\end{figure}
\newpage

\subsection{Phasendiagramm}
Die drei Phasen die im alltäglichem Leben beobachtet werden 
können sind, Gasförmig, Flüssig und Fest. Im Gasförmigen Zustand 
haben die jeweiligen Teilchen genug Energie um Wechselwirkungen 
zwischeneinander zu durchbrechen und sich frei im Raum zu 
bewegen. In der Flüssigen Phase haben die Teilchen nicht mehr 
genug Energie um die gegenseitigen Wechselwirkungne zu 
überwinden, sie haben aber immernoch genug Energie übrig um 
sich innerhalb der Flüssigkeit frei zu bewegen. In der festen 
Phase haben die Teilchen kaum noch Energie übrig weswegen sie 
sich in einem festen Gitter an ordnen und nur noch an ihrem 
festen Platz schwingen können. 
\begin{figure}[H]
\centering
\includegraphics[scale = 1]{Bilder/Phasendiagramm Wasser.png}
\caption{Das p,T-Phasendiagramm von Wasser, in diesem Versuch 
wird die Siedepunktskurve oder auch Dampfdruckkurve gennant 
betrachtet\cite{Phasendiagrammwasser}}
\end{figure}

Die im Versuch zu betrachtende Dampfdruckkurve ist die Verbindung 
zwischen Gasförmig und Flüssig, genau auf der Kurve befinden 
sich die beiden Phasen in einem Thermodynamischen Gleichgewicht 
welches durch das Gibbs Potential ausgedrückt werden kann.
\begin{equation}
    \Delta G = U + p \cdot V - S\cdot T
\end{equation}
Diese Formel gibt das Potential der Teilchen an und kann ebenfalls als,
\begin{equation}
    \Delta G = \Delta \left(U + p \cdot V - S\cdot T\right) = 0
\end{equation}
dargestellt werden. Diese Formel bezeichnet dass das Potential des ,in diesem Fall 
,Wassers immer so niedrig wie möglich ist. Wenn dieses Gleichgewicht, 
also $\Delta G = 0$ nun also auf der Dampfdruckkurve liegt können 
sogar beide Phasen gleichzeitig existieren. Auf dem Triple-Punkt 
existieren sogar alle drei Phasen nebeneinander.

\newpage
\subsection{Clausius-Clapeyron-Gleichung}

Mithilfe der Clausius-Clapeyron-Gleichung lässt sich der Zusammenhang 
zwischen dem Dampfdruck und der Temperatur eines Systems ermitteln. 
Sie kann mithilfe einiger Annahmen aus der Formel des Carnot-Kreisprozesses 
hergeleitet werden. Hierzu wird die Formel des Wirkungsgrad $\eta$ des Carnot-Kreisprozesses 
verwendet, bei welcher die Übergänge des Kreisprozesses beachtet werden 
bei denen Arbeit verrichtet wird, dadurch erhählt man für den Wirkungsgrad 
$\eta$ 
\begin{equation}
    \eta = \frac{\Delta W}{\Delta Q} = \frac{\Delta p_D \cdot \left(V_D-V_{Fl}\right)}{r\left(T\right)} = \frac{\Delta T}{T + \Delta T}
\label{Carnot-Kreisprozess}
\end{equation}
durch umformen der Formel \ref{Carnot-Kreisprozess} ergibt sich
\begin{equation}
    \frac{\Delta \textit{p}_D}{\Delta T} = \frac{r\left(T\right)}{\left(V_D-V_{Fl}\right)\cdot \left(T + \Delta T\right)}
\label{Carnot-Umgestellt}
\end{equation}
wenn nun angenommen wird das $\Delta T$ gegen 0 geht kann Gleichung 
\ref{Carnot-Umgestellt} zur Clausius-Clapeyron-Gleichung angepasst 
werden.
\begin{equation}
    \frac{d p_D}{d T} = \frac{r\left(T\right)}{\left(V_D - V_{Fl}\right) \cdot T}
\label{Clausius-Clapeyron-Gleichung}
\end{equation}
Um nun den Dampfdruck in Abhängigkeit der Temperatur zu bekommen wird 
angenommen, dass VD >> VF l und das $r\left(T\right)$ im betrachteten Temperaturbereich Konstant ist,
dementsprechend $r\left(T\right) = r$. Diese annahmen zusammen mit Gleichung \ref{Clausius-Clapeyron-Gleichung} ergeben,
\begin{equation}
    \frac{d p_D}{dT} = \frac{r}{V_D \cdot T}
    \label{DampfdruckTemp}
\end{equation}
Um nun aus Gleichung \ref{DampfdruckTemp} eine Formel nur für den Dampfdruck 
zu erhalten, wird angenommen dass das Dampfvolumen dass eines Idealen 
Gases entspricht, zusätzlich wird noch mithilfe von Trennung der Variablen 
Integriert. Dadurch erhält man für $d_p$
\begin{equation}
    p_D = p_D^0 \cdot exp\left[-\frac{r}{R}\cdot \left(\frac{1}{T}-\frac{1}{T_0}\right)\right]
\end{equation}
Hierbei ist $p_D$ der neue Dampfdruck, $P_D^0$ der Dampdruck des 
Anfangszustand, r die Verdampfungswärme, R die allgemeine Gaskonstante, 
T die neue Temperatur und $T_0$ die Anfangs Temperatur.

\newpage

\subsection{Flüssigkeitsbarometer}
Flüssigkeitsbarometer bestehen aus einem am Oberen Ende geschlossenem 
Rohr welches bis auf eine kleine Menge an Luft mit einer Flüssigkeit 
gefüllt sind. Da diese Flüssigkeit nun durch die Gravitation 
nach unten gezogen wird entsteht ein Kräftegleichgewicht und 
die Flüssigkeit wird im Rohr gehalten. Damit nun bei Druck Änderungen 
außerhalb des Rohres die Füllhöhe steigen kann wird außerhalb 
ein nach oben geöffnetes Reservoir angebracht damit ein 
Druckausgleich stattfinden kann. Flüssigkeitsbarometer 
verwenden meist Quecksilber als Flüssigkeit da Quecksilber eine 
sehr hohe dichte hat und das Barometer dadurch kompakt gehalten 
werden kann. Außerdem verdunstet Quecksilber nur sehr langsam 
weswegen Quecksilberbarometer sehr lange benutzt werden können 
bevor sie erneuert werden müssen.
\begin{figure}[H]
    \centering
    \includegraphics{Bilder/Flüssigkeitsbarometer.png}
    \caption{Die Schematische Darstellung eines Flüssigkeitsbarometer \cite{Flüssigkeitsbarometer}}
\end{figure}

