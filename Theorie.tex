\section{Theoretische Grundlagen}

\subsection{Wasserstrahlpumpe}
Die im Versuch verwendete Wasserstrahlpumpe funktioniert nicht 
wie eine Verdrängungspumpe, bei der Verdrängunspumpe gibt es 
Hauptsächlich vier Arbeitsschritte. 1. Ansaugen 2. Transportieren 
und Verdichten 3. Ausschieben. Im ersten Schritt wird aus dem 
zu entleerenden Volumen Luft angesaugt, die daraufhin im 
zweiten Schritt verdichtet und zum Ausgabepunkt transportiert 
wird, dort wird sie dann ausgeschoben, daraufhin wiederholt 
sich der Prozess nun. Bei der Wasserstrahlpumpe im Gegensatz 
beruht die Funktionsweise auf der Kontinuitätsgleichung. Das durch ein Rohr strömende Wasser wird an einer Verengung im Rohr beschleunigt. Aufgrund der Kontinuitätsgleichung ist die Fließgeschwindigkeit des Wassers proportional zur Verringerung des Rohrquerschnitts. Nach der Bernoulli-Gleichung führt dies an dieser Stelle zu einem Anstieg des dynamischen Drucks, während der statische Druck gleichzeitig abnimmt.

Der so entstehende Unterdruck ermöglicht das Entziehen von Luft aus dem zu evakuierenden System. Die angesaugte Luft wird in den Wasserstrahl eingemischt und zusammen mit diesem in den Außenraum abgeführt.
\begin{figure}[H]
    \centering
    \includegraphics[height = 10cm]{Bilder/Darstellung Wasserstrahlpumpe.png}
    \caption{Eine Schematische Darstellung der Wasserstrahlpumpe\cite{Wasserstrahlpumpe}}
    
    
\end{figure}
\newpage

\subsection{Phasendiagramm}
Die drei Phasen, die im alltäglichen Leben beobachtet werden können, sind 
gasförmig, flüssig und fest. Im gasförmigen Zustand haben die jeweiligen 
Teilchen genug Energie, um Wechselwirkungen untereinander zu durchbrechen 
und sich frei im Raum zu bewegen. In der flüssigen Phase haben die Teilchen 
nicht mehr genug Energie, um die gegenseitigen Wechselwirkungen zu überwinden, 
sie haben aber immer noch genug Energie übrig, um sich innerhalb der 
Flüssigkeit frei zu bewegen. In der festen Phase haben die Teilchen kaum noch 
Energie übrig, weshalb sie sich in einem festen Gitter anordnen und nur 
noch an ihrem festen Platz schwingen können. Die Besonderheit von Wasser 
ist dass es bei 4°C eine Dichte anomalie gibt, dass bedeutet dass dort 
die Dichte von Wasser am höchsten ist und bei niedriger und höherer 
Temperatur die Dichte abnimmt, diese Eigenschaft kann man auch im 
Phasendiagramm erkennen denn durch diese Eigenschaft ist die 
Schmelz/Gefrierkurve stärker gebogen als bei normalen Stoffen.
\begin{figure}[H]
\centering
\includegraphics[scale = 1]{Bilder/Vergleich Phasendiagramm.png}
\caption{Das p,T-Phasendiagramm von von einem Stoff ohne Dichte anomalie 
und von Wasser mit Dichte Anomalie\cite{Phasendiagrammwasser}}
\end{figure}

Die im Versuch zu betrachtende Dampfdruckkurve ist die Verbindung zwischen gasförmig und flüssig. Genau auf der Kurve befinden sich die beiden Phasen in einem thermodynamischen Gleichgewicht, welches durch das Gibbs-Potential ausgedrückt werden kann.
\begin{equation}
    \Delta G = U + p \cdot V - S\cdot T
\end{equation}
Diese Formel gibt das Potential der Teilchen an und kann ebenfalls als
\begin{equation}
    \Delta G = \Delta \left(U + p \cdot V - S\cdot T\right) = 0
\end{equation}
dargestellt werden. Diese Formel bedeutet, dass das Potential des — in diesem Fall — Wassers immer so niedrig wie möglich ist. Wenn dieses Gleichgewicht, also $\Delta G = 0$, auf der Dampfdruckkurve liegt, können sogar beide Phasen gleichzeitig existieren. Am Tripelpunkt existieren sogar alle drei Phasen nebeneinander.

\newpage
\subsection{Clausius-Clapeyron-Gleichung}

Mithilfe der Clausius-Clapeyron-Gleichung lässt sich der Zusammenhang zwischen dem Dampfdruck und der Temperatur eines Systems ermitteln. Sie kann mithilfe einiger Annahmen aus der Formel des Carnot-Kreisprozesses hergeleitet werden. Hierzu wird die Formel für den Wirkungsgrad $\eta$
des Carnot-Kreisprozesses verwendet, bei welcher die Übergänge des Kreisprozesses betrachtet werden, bei denen Arbeit verrichtet wird. Dadurch erhält man für den Wirkungsgrad $\eta$:
\begin{equation}
    \eta = \frac{\Delta W}{\Delta Q} = \frac{\Delta p_D \cdot \left(V_D-V_{Fl}\right)}{r\left(T\right)} = \frac{\Delta T}{T + \Delta T}
\label{Carnot-Kreisprozess}
\end{equation}
durch umformen der Formel \ref{Carnot-Kreisprozess} ergibt sich
\begin{equation}
    \frac{\Delta \textit{p}_D}{\Delta T} = \frac{r\left(T\right)}{\left(V_D-V_{Fl}\right)\cdot \left(T + \Delta T\right)}
\label{Carnot-Umgestellt}
\end{equation}
wenn nun angenommen wird das $\Delta T$ gegen 0 geht kann Gleichung 
\ref{Carnot-Umgestellt} zur Clausius-Clapeyron-Gleichung angepasst 
werden.
\begin{equation}
    \frac{d p_D}{d T} = \frac{r\left(T\right)}{\left(V_D - V_{Fl}\right) \cdot T}
\label{Clausius-Clapeyron-Gleichung}
\end{equation}
Um nun den Dampfdruck in Abhängigkeit der Temperatur zu bekommen, wird 
angenommen dass VD >> VF l und das $r\left(T\right)$ im betrachteten Temperaturbereich Konstant ist,
dementsprechend $r\left(T\right) = r$. Diese annahmen zusammen mit Gleichung \ref{Clausius-Clapeyron-Gleichung} ergeben,
\begin{equation}
    \frac{d p_D}{dT} = \frac{r}{V_D \cdot T}
    \label{DampfdruckTemp}
\end{equation}
Um nun aus Gleichung \ref{DampfdruckTemp} eine Formel nur für den Dampfdruck 
zu erhalten, wird angenommen dass das Dampfvolumen dass eines Idealen 
Gases entspricht, zusätzlich wird noch mithilfe von Trennung der Variablen 
Integriert. Dadurch erhält man für $d_p$
\begin{equation}
    p_D = p_D^0 \cdot exp\left[-\frac{r}{R}\cdot \left(\frac{1}{T}-\frac{1}{T_0}\right)\right]
\end{equation}
Hierbei ist $p_D$ der neue Dampfdruck, $P_D^0$ der Dampdruck des 
Anfangszustand, r die Verdampfungswärme, R die allgemeine Gaskonstante, 
T die neue Temperatur und $T_0$ die Anfangs Temperatur.

\newpage

\subsection{Flüssigkeitsbarometer}
Flüssigkeitsbarometer bestehen aus einem am oberen Ende geschlossenen Rohr, das bis auf eine kleine Menge Luft mit einer Flüssigkeit gefüllt ist. Da diese Flüssigkeit nun durch die Gravitation nach unten gezogen wird, entsteht ein Kräftegleichgewicht und die Flüssigkeit wird im Rohr gehalten. Damit bei Druckänderungen außerhalb des Rohres die Füllhöhe steigen kann, wird außen ein nach oben geöffnetes Reservoir angebracht, damit ein Druckausgleich stattfinden kann. Flüssigkeitsbarometer verwenden meist Quecksilber als Flüssigkeit, da Quecksilber eine sehr hohe Dichte hat und das Barometer dadurch kompakt gehalten werden kann. Außerdem verdunstet Quecksilber nur sehr langsam, weshalb Quecksilberbarometer sehr lange benutzt werden können, bevor sie erneuert werden müssen.
\begin{figure}[H]
    \centering
    \includegraphics{Bilder/Flüssigkeitsbarometer.png}
    \caption{Die Schematische Darstellung eines Flüssigkeitsbarometer \cite{Flüssigkeitsbarometer}}
\end{figure}


\begin{equation}
p_D = p_{0D} \, \exp\left( -\frac{r}{R} \left( \frac{1}{T} - \frac{1}{T_0} \right) \right)
\label{eq:pd}
\end{equation}
