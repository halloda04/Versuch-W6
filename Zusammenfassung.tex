\section{Zusammenfassung}

Die zentrale Erkenntnis des Versuchs, die Abhängigkeit des Siedepunkts von Wasser vom Umgebungsdruck sowie von der Temperatur, konnte bestätigt werden. Die gemessene Dampfdruckkurve stimmt weitestgehend mit der theoretischen Dampfdruckkurve überein. Auch die materialspezifische Konstante der molaren Verdampfungswärme von Wasser, welche z.B. auch für den Versuch über die Siedepunkterhöhung wichtig ist, konnte nahe am Literaturwert bestimmt werden. Die Abweichungen von den Literaturwerten können auf verschiedene Fehlerquellen zurückgeführt werden: Zum einen die Subjektivität bei der Bestimmung des Siedepunkts, zum anderen Ablesefehler bei der Quecksilbersäule. Letzteres könnte mit einem genaueren Messgerät für den Druck behoben werden. Um den Siedepunkt objektiv zu messen, ist wahrscheinlich ein anderer Versuchsaufbau erforderlich.