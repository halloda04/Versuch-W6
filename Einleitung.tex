\section{Einleitung}

Die Phasenübergänge von Wasser sind ein wichtiger Bestandteil 
unseres täglichen Lebens und somit auch der Phasenübergang 
Wassers von Flüssig zu Gasförmig. Dieser Phasenübergang findet 
sich im Alltag, in der Technik und in der Forschung. In der 
Technik kann er dazu verwendet werden um Mikro-Prozessoren zu 
kühlen, diese können auf sehr kleinem Raum viel Wärme abgeben, 
damit sie dann immernoch effizient laufen können werden 
Wärmerohre verwendet die diese entstehende Wärme abtransportieren 
können. Die Rohre funktionieren auf der Basis dass eine 
Flüssigkeit auf der einen Seite verdampft und auf der anderen 
Seite wieder kondensiert und somit die Verdampfungsenergie zum 
zweiten Punkt bringt. Damit nun aber nicht der Verdampfunspunkt 
wie zum Beispiel bei Wasser nicht bei 100°C liegt, wird der 
Druck innerhalb des Rohres gesenkt und der Verdampfungspunkt 
sinkt ebenso.  (Quelle)

Der Versuch ist dem Bereich der Wärmelehre zuzuorden.