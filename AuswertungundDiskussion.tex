\section{Auswertung}
\subsection{Umgebungsdruck}
Es wurde für die Quecksilbersäule eine Höhe von 953,0mb und für die Kuppel eine Höhe von 1,2 mb gemessen. Die Raumtemperatur war $(21,1 \pm0,1)$°C. Für die Geographische Lage von Augsburg wurde eine Höhenlage von 500m, eine östliche Länge von 10° 52min und eine nördliche Breite von 48° 21min verwendet. Mit diesen Werten lässt sich die gemessene Quecksilbersäure korrigieren. 

\begin{table}[H]
\centering
\begin{tabular}{c|c}

Gemessener Druck & 953,0mb \\
\hline
Temperaturkorrektur(21°C) & $- 3,64$mb \\

Kuppenkorrektur(1,2mb) & +0,64mb\\

Höhenkorrektur(500m) &  $ -0,09$mb \\

Breitenkorrektur(48°) &  $ +0,39$mb \\
\hline
Korregierter Druck $p_0$ & = 950,30mb \\

\end{tabular}
\caption{korregierter Druck in mbar}
\end{table}


Es Ergibt sich ein wahrer Druck von $p_0 = 950,30mb = 95030$ Pa. Hier wird angenommen das keine weiteren Fehler gemacht wurden und somit Auf die Fehlerrechnung versichtet. Die Steighöhe der Quecksilbersäule ist im folgendem

\begin{table}[H]
\centering
\begin{tabular}{c|c}
$h_{Hg}$ [mm] & T[°C] \\
\hline
645 $\pm$ 3 & 42.2 $\pm$ 0.05 \\
627 $\pm$ 3 & 50.9 $\pm$ 0.05 \\
590 $\pm$ 3 & 56.9 $\pm$ 0.05 \\
581 $\pm$ 3 & 58.7 $\pm$ 0.05 \\
555 $\pm$ 3 & 61.7 $\pm$ 0.05 \\
548 $\pm$ 3 & 63.5 $\pm$ 0.05 \\
536 $\pm$ 3 & 64.5 $\pm$ 0.05 \\
522 $\pm$ 3 & 66.8 $\pm$ 0.05 \\
508 $\pm$ 3 & 68.2 $\pm$ 0.05 \\
490 $\pm$ 3 & 70.3 $\pm$ 0.05 \\
473 $\pm$ 3 & 72.5 $\pm$ 0.05 \\
453 $\pm$ 3 & 75.0 $\pm$ 0.05 \\
438 $\pm$ 3 & 76.0 $\pm$ 0.05 \\
420 $\pm$ 3 & 78.1 $\pm$ 0.05 \\
400 $\pm$ 3 & 79.2 $\pm$ 0.05 \\
373 $\pm$ 3 & 81.2 $\pm$ 0.05 \\
350 $\pm$ 3 & 83.0 $\pm$ 0.05 \\
328 $\pm$ 3 & 84.3 $\pm$ 0.05 \\
305 $\pm$ 3 & 86.0 $\pm$ 0.05 \\
290 $\pm$ 3 & 86.7 $\pm$ 0.05 \\
260 $\pm$ 3 & 88.0 $\pm$ 0.05 \\
238 $\pm$ 3 & 89.0 $\pm$ 0.05 \\
170 $\pm$ 3 & 92.7 $\pm$ 0.05 \\
\end{tabular}
\caption{Die Messwerte von der Quecksilbersäule und der zugehörigen Temperatur mit Ablesefehlern}
\label{tb:messwerte}
\end{table}

gegen die ensprechende Temperatur aufgetragen. Durch die Differenz von dem Außendruck und dem Druck im Gefäß. Um Damit den Dampfdruck $p_D$ zu berechnen. Betrachtet man zu erst den Druck $p$

\begin{equation}
p = \frac{F}{A}
\end{equation}

Da die relevante Kraft hier die Gravitationskraft ist folgt mit $F = F_g$ daraus:

\begin{equation}
p = \frac{m_{Hg} \cdot g}{A_{Rohr}} = \frac{\rho_{Hg} \cdot A_{Rohr} \cdot h \cdot g}{A_{Rohr}} = \rho_{Hg} \cdot g \cdot h = p_0 - p_D
\end{equation}

Damit folgt für den Dampfdruck 
\begin{equation}
p_D = p_0 - p_{Hg} \cdot g \cdot h
\label{eq:d_P2_FR}
\end{equation}

Der Fehler ist gegeben durch wobei der Fehler $\Delta h = 3$mm von dem Ablesefehler der Höhe der Quecksilbersäule kommt.

\begin{equation}
\Delta p_D = \pm \rho_{Hg} \cdot g \cdot \Delta h
\label{eq:d_P2}
\end{equation}

Da $g = 9,81 \frac{\text{m}}{\text{s}^2}$ und $p_D = 95030$ Pa wie oben erläutert als fehlerfrei angenommen wurden. $P_{Hg}= 13534 \frac{\text{kg}}{\text{m}^3}$ ist ein Literaturwert \cite{Hgdichte} und wurde auch als fehlerfrei angenommen. 


\begin{table}[H]
\centering
\begin{tabular}{r | r}
T [°C] & $P_D$ [Pa] \\
\hline
42.2 $\pm$ 0.05 & 9318.4 $\pm$ 400 \\
50.9 $\pm$ 0.05 & 11710.3 $\pm$ 400 \\
56.9 $\pm$ 0.05 & 16627.1 $\pm$ 400 \\
58.7 $\pm$ 0.05 & 17823.1 $\pm$ 400 \\
61.7 $\pm$ 0.05 & 21278.1 $\pm$ 400 \\
63.5 $\pm$ 0.05 & 22208.3 $\pm$ 400 \\
64.5 $\pm$ 0.05 & 23803.0 $\pm$ 400 \\
66.8 $\pm$ 0.05 & 25663.4 $\pm$ 400 \\
68.2 $\pm$ 0.05 & 27523.8 $\pm$ 400 \\
70.3 $\pm$ 0.05 & 29915.7 $\pm$ 400 \\
72.5 $\pm$ 0.05 & 32174.8 $\pm$ 400 \\
75.0 $\pm$ 0.05 & 34832.5 $\pm$ 400 \\
76.0 $\pm$ 0.05 & 36825.8 $\pm$ 400 \\
78.1 $\pm$ 0.05 & 39217.8 $\pm$ 400 \\
79.2 $\pm$ 0.05 & 41875.5 $\pm$ 400 \\
81.2 $\pm$ 0.05 & 45463.4 $\pm$ 400 \\
83.0 $\pm$ 0.05 & 48519.8 $\pm$ 400 \\
84.3 $\pm$ 0.05 & 51443.3 $\pm$ 400 \\
86.0 $\pm$ 0.05 & 54499.7 $\pm$ 400 \\
86.7 $\pm$ 0.05 & 56493.0 $\pm$ 400 \\
88.0 $\pm$ 0.05 & 60479.6 $\pm$ 400 \\
89.0 $\pm$ 0.05 & 63403.1 $\pm$ 400 \\
92.7 $\pm$ 0.05 & 72439.3 $\pm$ 400 \\
\end{tabular}
\caption{Die mit Hilfe von Gleichung (\ref{eq:d_P2_FR}) und den Werten aus Tabelle \ref{tb:messwerte} erechneten Dampfdrücke mit Fehlern gemäß Gleichung \ref{eq:d_P2}} 
\end{table}

Wenn man diese Werte, wie im folgenden graphisch aufträgt,

\begin{figure}[H]
    \centering
    \includegraphics[height = 12cm]{Bilder/pDgegenT.png}
    \caption{Der Dampfdruck $P_D$ aufgetragen gegen die Temperatur T die Fitkurve wurde mittels Numpy.polyfit erstellt.}
    
\end{figure}

ist der exponentielle Zusammenhang, so wie  in Gleichung (\ref{eq:pd}) verausgesagt, zwischen Dampfdruck und Temperaturanstieg zu erkennen. NOCH VERGLEICH KURVE

\subsection{Lineare Dampfdruckkurve}

