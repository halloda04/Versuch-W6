\section{Auswertung}
\subsection{Umgebungsdruck}
Für die Quecksilbersäule wurde eine Höhe von 953,0~mbar und für die Kuppel eine Höhe von 1,2~mbar gemessen. Die Raumtemperatur betrug $(21,1 \pm 0,1)~^\circ\mathrm{C}$. 
Für die geographische Lage von Augsburg wurden eine Höhenlage von 500~m, eine östliche Länge von 10°\,52' und eine nördliche Breite von 48°\,21' verwendet. 
Mit diesen Werten lässt sich die gemessene Quecksilbersäule entsprechend korrigieren.


\begin{table}[H]
\centering
\begin{tabular}{c|c}

Gemessener Druck & 953,0mb \\
\hline
Temperaturkorrektur(21°C) & $- 3,64$mb \\

Kuppenkorrektur(1,2mb) & +0,64mb\\

Höhenkorrektur(500m) &  $ -0,09$mb \\

Breitenkorrektur(48°) &  $ +0,39$mb \\
\hline
Korregierter Druck $p_0$ & = 950,30mb \\

\end{tabular}
\caption{korrigierter Druck in mbar}
\end{table}


Damit ergibt sich ein wahrer Atmosphärendruck von $p_0 = 950,30~\mathrm{mbar} = 95030~\mathrm{Pa}$. 
Hier wird angenommen, dass keine weiteren Fehler aufgetreten sind, und daher wird auf eine Fehlerrechnung verzichtet. 
Die Steighöhe der Quecksilbersäule ist im Folgenden angegeben.


\begin{table}[H]
\centering
\begin{tabular}{c|c}
$h_\mathrm{Hg}$ [mm] & $T$ [K] \\
\hline
645 $\pm$ 3 & 315.35 $\pm$ 0.05 \\
627 $\pm$ 3 & 324.05 $\pm$ 0.05 \\
590 $\pm$ 3 & 330.05 $\pm$ 0.05 \\
581 $\pm$ 3 & 331.85 $\pm$ 0.05 \\
555 $\pm$ 3 & 334.85 $\pm$ 0.05 \\
548 $\pm$ 3 & 336.65 $\pm$ 0.05 \\
536 $\pm$ 3 & 337.65 $\pm$ 0.05 \\
522 $\pm$ 3 & 339.95 $\pm$ 0.05 \\
508 $\pm$ 3 & 341.35 $\pm$ 0.05 \\
490 $\pm$ 3 & 343.45 $\pm$ 0.05 \\
473 $\pm$ 3 & 345.15 $\pm$ 0.05 \\
453 $\pm$ 3 & 348.15 $\pm$ 0.05 \\
438 $\pm$ 3 & 349.15 $\pm$ 0.05 \\
420 $\pm$ 3 & 351.25 $\pm$ 0.05 \\
400 $\pm$ 3 & 352.35 $\pm$ 0.05 \\
373 $\pm$ 3 & 354.35 $\pm$ 0.05 \\
350 $\pm$ 3 & 356.15 $\pm$ 0.05 \\
328 $\pm$ 3 & 357.45 $\pm$ 0.05 \\
305 $\pm$ 3 & 359.15 $\pm$ 0.05 \\
290 $\pm$ 3 & 359.85 $\pm$ 0.05 \\
260 $\pm$ 3 & 361.15 $\pm$ 0.05 \\
238 $\pm$ 3 & 362.15 $\pm$ 0.05 \\
170 $\pm$ 3 & 365.85 $\pm$ 0.05 \\
\end{tabular}
\caption{Die Messwerte von der Quecksilbersäule und der zugehörigen Temperatur mit Ablesefehlern}
\label{tb:messwerte}
\end{table}

gegen die entsprechende Temperatur aufgetragen. Bei dieser Messung gibt es einige Fehlerquellen. 
Die wohl größte ist die Bestimmung des Zeitpunkts, an dem das Wasser zu kochen beginnt. 
Dieser wurde bei der Durchführung als der Zeitpunkt definiert, an dem am gesamten Rand des Gefäßes Luftblasen zu sehen waren, sich also an mehreren Siedesteinchen Blasen bilden. 
Trotz dieser Definition muss dieser Zeitpunkt immer noch subjektiv festgestellt werden. 
Die Temperatur und der Druck müssen dann zeitgleich abgelesen werden. Beides ist fehleranfällig. 
Eine weitere Fehlerquelle ist das Ablesen der Höhe der Quecksilbersäule. Hier wird zum einen die Höhe der Kuppel nicht berücksichtigt, zum anderen muss die Höhe an einer Millimeterskala abgelesen werden, was ebenfalls fehleranfällig ist. 
Aus diesen Gründen beträgt der Messfehler der Höhe $\pm 3~\mathrm{mm}$. 
Der Fehler der Temperatur entsteht dadurch, dass das Display nur eine Nachkommastelle anzeigen kann.\\
Durch die Differenz zwischen dem Außendruck und dem Druck im Gefäß lässt sich der Dampfdruck $p_D$ betrachten. 
Betrachtet man jedoch zuerst den Druck $p$.


\begin{equation}
p = \frac{F}{A}
\end{equation}

Da die relevante Kraft hier die Gravitationskraft ist, folgt mit $F = F_g$ daraus:

\begin{equation}
p = \frac{m_{Hg} \cdot g}{A_{Rohr}} = \frac{\rho_{Hg} \cdot A_{Rohr} \cdot h \cdot g}{A_{Rohr}} = \rho_{Hg} \cdot g \cdot h = p_0 - p_D
\end{equation}

Damit folgt für den Dampfdruck.
\begin{equation}
p_D = p_0 - p_{Hg} \cdot g \cdot h
\label{eq:d_P2_FR}
\end{equation}

Der Fehler ist gegeben durch.

\begin{equation}
\Delta p_D = \pm \rho_{Hg} \cdot g \cdot \Delta h
\label{eq:d_P2}
\end{equation}

Mit dieser Formel kann der Dampfdruck $P_D$ berechnet werden. Für die Erdbeschleunigung wurde $g = 9,81 \frac{\text{m}}{\text{s}^2}$ verwendet. Für den Außendruck $p_0 = 95030$ Pa. $P_{Hg}= 13534 \frac{\text{kg}}{\text{m}^3}$ ist ein Literaturwert \cite{Hgdichte}. In Tabelle \ref{tb:pd_K} sind die daraus Folgenden Dampfdrücke aufgetragen.


\begin{table}[H]
\centering
\begin{tabular}{r | r}
T [K] & $P_D$ [Pa] \\
\hline
315.35 $\pm$ 0.05 & 9318.4 $\pm$ 400 \\
324.05 $\pm$ 0.05 & 11710.3 $\pm$ 400 \\
330.05 $\pm$ 0.05 & 16627.1 $\pm$ 400 \\
331.85 $\pm$ 0.05 & 17823.1 $\pm$ 400 \\
334.85 $\pm$ 0.05 & 21278.1 $\pm$ 400 \\
336.65 $\pm$ 0.05 & 22208.3 $\pm$ 400 \\
337.65 $\pm$ 0.05 & 23803.0 $\pm$ 400 \\
339.95 $\pm$ 0.05 & 25663.4 $\pm$ 400 \\
341.35 $\pm$ 0.05 & 27523.8 $\pm$ 400 \\
343.45 $\pm$ 0.05 & 29915.7 $\pm$ 400 \\
345.65 $\pm$ 0.05 & 32174.8 $\pm$ 400 \\
348.15 $\pm$ 0.05 & 34832.5 $\pm$ 400 \\
349.15 $\pm$ 0.05 & 36825.8 $\pm$ 400 \\
351.25 $\pm$ 0.05 & 39217.8 $\pm$ 400 \\
352.35 $\pm$ 0.05 & 41875.5 $\pm$ 400 \\
354.35 $\pm$ 0.05 & 45463.4 $\pm$ 400 \\
357.15 $\pm$ 0.05 & 48519.8 $\pm$ 400 \\
357.45 $\pm$ 0.05 & 51443.3 $\pm$ 400 \\
359.15 $\pm$ 0.05 & 54499.7 $\pm$ 400 \\
359.85 $\pm$ 0.05 & 56493.0 $\pm$ 400 \\
361.15 $\pm$ 0.05 & 60479.6 $\pm$ 400 \\
362.15 $\pm$ 0.05 & 63403.1 $\pm$ 400 \\
365.85 $\pm$ 0.05 & 72439.3 $\pm$ 400 \\
\end{tabular}
\caption{Die mit Hilfe von Gleichung (\ref{eq:d_P2_FR}) und den Werten aus Tabelle \ref{tb:messwerte} erechneten Dampfdrücke mit Fehlern gemäß Gleichung (\ref{eq:d_P2})}
\label{tb:pd_K}
\end{table}


Wenn man diese Werte, wie im folgenden graphisch aufträgt,

\begin{figure}[H]
    \centering
    \includegraphics[height = 12cm]{Bilder/pDgegenT.png}
    \caption{Der Dampfdruck $P_D$ aufgetragen gegen die Temperatur T die Fitkurve wurde mittels Numpy.polyfit erstellt.}
    \label{gl:5}
\end{figure}

ist der exponentielle Zusammenhang, so wie  in Gleichung (\ref{eq:pd}) verausgesagt, zwischen Dampfdruck und Temperaturanstieg zu erkennen. Vergleicht man dies mit den theoretischen Werten für den Siedepunkt von 100°C bei Normaldruck, so erhält man mit Hilfe der Clausius-Clapeyron-Gleichung

\begin{equation}
p_D(T) = 101300~\mathrm{Pa} \cdot \exp \Bigg[ - \frac{40.642~\mathrm{kJ/mol}}{8.314~\mathrm{J/(mol \cdot K)}} \left( \frac{1}{T} - \frac{1}{373.15~\mathrm{K}} \right) \Bigg]
\end{equation}

Wenn man diese Werte in Abbildung \ref{gl:5} einträgt,

\begin{figure}[H]
    \centering
    \includegraphics[height = 12cm]{Bilder/Lieteraturwerte.png}
    \caption{Der Dampfdruck $P_D$ und der Theortische Druck aus Gleichung (13) aufgetragen gegen die Temperatur T die Fitkurve wurde mittels Numpy.polyfit erstellt.}
    
\end{figure}

kann man eine leichte Abweichung von den Messwerten, die mit steigenden Temperaturen zunimmt, beobachtet werden. Ein Grund könnte möglicherweise sein, dass die Steighöhe der Quecksilbersäule von der Temperatur oder Druck im System, noch wegen anderen Gründen abhängig ist.


\newpage
\subsection{Lineare Dampfdruckkurve}
Mit Hilfe des Logarithmus lässt sich ein Linearer Zusammenhang zwischen $P_D$ und $1/T$ herstellen. 

\begin{equation}
\ln(p_D) = \ln(p_D^{0}) 
- \frac{r}{R}\left( \frac{1}{T} - \frac{1}{T_0} \right)
= -\,\frac{r}{R}\cdot \frac{1}{T}
+ \frac{r}{R T_0}
+ \ln(p_D^{0}).
\label{eq:ln}
\end{equation}

Trägt man die Werte graphisch auf so errhält man Abbildung \ref{gf:ln}

\begin{figure}[H]
    \centering
    \includegraphics[height = 12cm]{Bilder/Reziproke.png}
    \caption{Der Logarithmus Dampfdruck $ln(P_D)$ aufgetragen gegen die Temperatur T in $\frac{1}{K}$ die Fitkurve wurde mittels Numpy.polyfit erstellt.}
    \label{gf:ln}
\end{figure}

Die Steigungen sind in Tabelle 4 aufgelistet, die Einheit der Steigung ist in [K] da der Logarithmus dimensionslos ist.

\begin{table}[H]
\centering
\begin{tabular}{r | r}
Gerade & Steigung [K] \\
\hline
Ausgleichsgerade & -4820 \\
Grenzgerade 1 & -4525\\
Grenzgerade 2 & -5133 \\

\end{tabular}
\caption{Die Steigung der Geraden} 
\end{table}

Damit kann man den Fehler der Steigung folgendermaßen berechnen:
\begin{equation}
\Delta m \pm \frac{m_2 - m_1}{2} = 304 \text{K}
\end{equation}

\subsection{Molare Verdampfungswärme}

Durch Differentiation und Umstellung von Gleichung (\ref{eq:ln}) ergibt sich für die molare Verdampfungswärme unter Verwendung der allgemeinen Gaskonstante $R = 8.314 \,\frac{\mathrm{J}}{\mathrm{mol\,K}}$

\begin{equation}
r = -m \cdot R \approx 40.07 \frac{\mathrm{KJ}}{\mathrm{mol}}
\end{equation}

Da $R$ als fehlerfrei angenommen wird, ergibt sich für den Größtfehler von $r$ folgende Gleichung:

\begin{equation}
\Delta r = \pm \left| \frac{\partial r}{\partial m} \right| \Delta m = \pm \Delta m \cdot R
\end{equation}

Nach Einsetzen der Werte ergibt sich aus den Messwerten die molare Verdampfungswärme für Wasser zu:

\begin{equation}
r \approx (40,07 \pm 2,47)~\mathrm{kJ/mol}
\end{equation}

Der Literaturwert aus \cite{Tipler} beträgt $r_L = 40,66~\mathrm{kJ/mol}$. 
Somit liegt die bestimmte Verdampfungswärme $r$ innerhalb der Fehlerschranke, die durch $\Delta r$ angegeben ist.